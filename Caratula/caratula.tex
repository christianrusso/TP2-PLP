\documentclass[11pt,a4paper]{article}

\usepackage[margin=0.5in, top=3cm, bottom=2cm]{geometry}
\usepackage[spanish, activeacute]{babel}
\usepackage[utf8]{inputenc}
\usepackage{amsthm}
\usepackage{amsmath}
\usepackage{amsfonts}
\usepackage{amssymb}
\usepackage{graphicx} %Para incluir el logo de la UBA
\usepackage{caratula} %Para armar el cuadro de integrantes
\usepackage{hyperref} %Para escribir urls
\usepackage{pdfpages} %Para incluir pdf's
\usepackage{todonotes}
\usepackage{float}
\usepackage[linesnumbered]{algorithm2e}


\graphicspath{{imagenes/}}

%Cosas para escribir codigo fuente
%Fuente: http://en.wikibooks.org/wiki/LaTeX/Source_Code_Listings
\usepackage{listings}
\usepackage{color}

\setcounter{secnumdepth}{5}

\begin{document}

\integrante{Gasperi Jabalera, Fernando}{56/09}{fgasperijabalera@gmail.com}
\integrante{Russo, Christian Sebasti'an}{679/10}{christian.russo8@gmail.com}
\integrante{Tagliavini Ponce, Guido}{783/11}{guido.tag@gmail.com}

\def\Materia{Paradigmas de Lenguajes de Programaci'on}
\def\Titulo{Trabajo Pr\'{a}ctico 2}
\def\Fecha{3 de noviembre de 2015}

%----- CARATULA -----%

\thispagestyle{empty}

\begin{center}
	\includegraphics[scale = 0.25]{imagenes/logo_uba.jpg}
\end{center}

\begin{center}
	{\textbf{\large UNIVERSIDAD DE BUENOS AIRES}}\\[1.5em]
	{\textbf{\large Departamento de Computaci\'{o}n}}\\[1.5em]
    {\textbf{\large Facultad de Ciencias Exactas y Naturales}}\\
    \vspace{35mm}
    {\LARGE\textbf{\Materia}}\\[1em]    
    \vspace{15mm}
    {\Large \textbf{\Titulo}}\\[1em]
    \vspace{15mm}
    {\textbf{\Large \Fecha}}\\
    \vspace{15mm}
	{\textbf{\Large Grupo: foldr foldr []
[`f`, `o`, `l`, `d`, `r`]}}\\
    \vspace{15mm}
    \textbf{\tablaints}
\end{center}

\newpage

\section{Comparaci'on de tiempos}

Armamos 10 casos de tests para medir el tiempo, que se pueden ver en el c'odigo fuente, y los ejecutamos con \texttt{todosConstruir1} y \texttt{todosConstruir2}. Comparamos el tiempo de ejecuci'on de cada predicado, para cada test.

\begin{figure}[H]
    \centering
    \includegraphics[width=7.0in]{imagenes/grafico.png}
    %\caption{Resultados}
    \label{simulationfigure}
\end{figure}

\noindent
Como se esperaba, la construcci'on de las soluciones utilizando programaci'on din'amica v'ia \texttt{todosConstruir2} fue m'as r'apida.

\end{document}
